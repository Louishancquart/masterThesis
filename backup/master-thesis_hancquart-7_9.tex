%% LyX 2.1.4 created this file.  For more info, see http://www.lyx.org/.
%% Do not edit unless you really know what you are doing.
\documentclass[11pt,a4paper,english,thesis]{dcsbook}
\usepackage[utf8]{inputenc}
\setcounter{secnumdepth}{3}
\setcounter{tocdepth}{4}
\usepackage{color}
\usepackage{babel}
\usepackage{float}
\usepackage{wrapfig}
\usepackage{textcomp}
\usepackage{graphicx}
\usepackage[unicode=true,pdfusetitle,
 bookmarks=true,bookmarksnumbered=true,bookmarksopen=true,bookmarksopenlevel=1,
 breaklinks=true,pdfborder={0 0 0},backref=false,colorlinks=true]
 {hyperref}
\hypersetup{
 urlcolor=linkcolor,linkcolor=linkcolor,citecolor=linkcolor}
\usepackage{breakurl}

\makeatletter

%%%%%%%%%%%%%%%%%%%%%%%%%%%%%% LyX specific LaTeX commands.
\special{papersize=\the\paperwidth,\the\paperheight}


%%%%%%%%%%%%%%%%%%%%%%%%%%%%%% Textclass specific LaTeX commands.
\RequirePackage{dcslib}[2012/01/30]

%%%%%%%%%%%%%%%%%%%%%%%%%%%%%% User specified LaTeX commands.
%
%  $Id: thesis-template.lyx,v 1.7 2011/12/22 12:10:18 sobaniec Exp $
%

\makeatother

\usepackage{listings}
\lstset{basicstyle={\ttfamily},
commentstyle={\color{green}\ttfamily},
keywordstyle={\color{blue}\ttfamily},
language={Java},
morecomment={[l][\color{magenta}]{\#}},
numbers=left,
numberstyle={\tiny},
stringstyle={\color{red}\ttfamily},
backgroundcolor={\color{verylightgrey}},
breakautoindent=true,
xleftmargin={10pt},
breaklines=true,
breakatwhitespace=true}
\renewcommand{\lstlistingname}{\inputencoding{latin9}Listing}

\begin{document}

\author{Louis Hancquart}


\title{<the name of the system>: Design and implementation of a decentralized
application for ad-free video hosting}


\date{Poznań, 2016}


\supervisor{Paweł T. Wojciechowski}

\maketitle


\frontmatter

%%code background color
\definecolor{verylightgrey}{RGB}{217,217,217} 

\tableofcontents{}


\chapter*{Abstract}

lroem ipsum .............sd;lfks;dlkf;sldkf;lskd;flks;dlkf;

\mainmatter


\chapter{Introduction}


\paragraph{The goal and the scope of the thesis}

As a continuum from the time where TV was set in the center of the
living-room, people are watching more and Films and videos. it is
set that the watching time per day didn't stop increasing from 20
years (source) . And it is not about to stop. The next generation
of video-eaters are even more hungry. They grown up with a smartphone
and a lot of video applications available everywhere, all the time.
The success of applications like hangouts, youtube, facebook ,vimeo...
demonstrate an increasing usage and demand of videos.

If the video consumption is constantly increasing for years, the medium
is changing. Traditionally, the TV is a top-down media type. There
is an antenna owned by companies or governments channels that emits
programs towards people watching. It is a one-to-many structure. On
the Internet, the topology is totally different : born as Arpanet
, a university network to exchange research data, the Internet nature
is to be flat. It is a one-to-one connection ( or peer to peer ).
This medium, firstly reserved reserved to experts, has grown and got
simpler to use, then spread to the majority of people. Nowadays almost
everyone has an Internet connection. As a result, more and more people
are switching off their TVs to browse through the Internet. Some content
provider like Youtube or Dailymotin rose, reproducing the previous
one-to-many pattern over the Internet. But as it makes no difference
to increase the audience of a TV channel using an antenna, it is a
bigger problem on a mesh network. 

By nature, Internet is a flat mesh network where each node can emit
the content he likes. Transposing old technologies like TV on the
Internet suppose to concentrate the majority of the data flux on a
few nodes ( e.g. Youtube, Facebook, etc.) . Of course, watching videos
on central websites like Youtube is by far a different experience
than watching videos on TV. Many people can provide their content
easily. It changes a lot of things : a lot more content providers,
an infinite channels choice, an interaction between the provider and
the audience. Nevertheless, on the most popular video platforms, all
the content produce by the providers is the property of the host,
e.g. the platform. The free access to the content provided is not
a mandatory and may not remain as it is nowadays. The content control
on some videos has already been observed and to support the service,
providers sell advertisement space to their platform. In the future,
the Internet video possibilities will look like its ancestor: the
TV: central video supply , cut with advertisement and content controlled
moreover in an environment meant to have a flat topology.

It is a problem that most of people doesn't get for the moment. Big
companies already got the almost-monopoly of the video delivery on
Internet. Simulating the Internet flat nature and freedom of speaking,
they are loosing a lot of money storing freely a lot of videos from
everyone. They are probably waiting to get more audience than the
TV. 

So, What can we do to avoid this? One solution is the peer to peer
: Content hosted through by everyone for everyone. This technology
is now available for more than a decade. At first it was the only
way to exchange videos on the Internet. This network had too small
data traffic capacities to allow video downloading from one website
as we are doing right now. But recently this technology loose in popularity
as it has begun too complex to use for the lambda user switching from
TV to the Internet. 

The aim of Thesis is to provide an efficient application to manage
torrents automatically and to free the end user from downloading ,
latency , etc.. 

And I will go through theses points: Background , ...... .


\chapter{Background }

To begin, I will set the background context, necessary to understand
this thesis.

At first, I will draw your interest on the Internet Economical environment.
Then I will set the technical on which this thesis started on. Finally,
I will center the reflection on the particular problem this thesis
try to answer : decentralization and ad free video hosting application.


\section{Economical Context}

{*}In order to set up the context of a decentralized video application,
I would like to focus on 3 points : the network , the market, and
the different actors to take into account. 


\subsection{Actors}

On the web, there is different ways to get video content: video hosting
websites, peer-to-peer, streaming websites, direct download. I am
going to describe them here. 


\subsubsection{video hosting websites}

The video hosting services allows individual end users to upload and
share personal, business, or royalty-free videos and to watch them
legally.\cite{key-9}. We spend more and more time on Intertnet and
video websites. Traffic on video hosting services are rising from
a few years and it is slowly replacing the TV. 

There is a few reasons that makes these websites so popular. First,
since internet has a flat topology, the communication is not top-down
but more peer to peer. It gave a will of expressiveness. ( source
to find ). Second of all, most people do not own web servers, and
this has created demand for user-generated video content hosting.
The hosting websites definitely answer this demand, providing a user-friendly
interface and a common place to upload and watch videos. Further more,
the video hosting websites are suiting the user expression will are
they are adaped to low quality video like phone video, etc. It might
change with the growing professionalism and quality video format of
creators. 

( https://en.wikipedia.org/wiki/Video\_hosting\_service )


\paragraph{Usage }

Users generally will upload via the hosting service's website, mobile
or desktop applications or APIs. The type of video content uploaded
can be anything from short video clips and digital audio files all
the way to full-length TV shows and movies. The video host stores
the video on its server offers individual different types of embed
codes or links to allow others to view this video. The website, mainly
used as the video hosting website, is usually called the video sharing
website. Some video hosting services offer copyrighted TV shows, movies
and music for which they have not obtained the necessary copyright
approvals.


\paragraph{History}

Nowadays, there is a lot of video hosting websites : Youtube, Vimeo
, Dailymotion, are one of the most famous example. The first website
of the kind was founded in was founded in 1997 by Chase Norlin\cite{shareyourworld.com}.
Similar to the actual video hosting websites, shareyourworld.com allowed
users to upload clips or full videos in different file formats. The
site shutdown in 2001 due to budget and bandwidth problems.


\paragraph{centralisation}

One caracteristic of the video hosting platforms is to be centralized.
It is a website on wihch every user connects and request a video from
this one. 


\subparagraph{network issues}

Thousands of billions of peopole are connecting to the same node of
the network. It makes a huge load of requests and content supply.
The servers has to uses strategies like load ballancing to handle
this and serve the videos. 

Internet is a mesh network and distributting most of the traffic on
a few node is difficult to support for the servers and the network
equipment. ISPs( ) are fighting agains such big video hosting services
like youtube to make them pay a tax for using their network. As we
said earlier, it was even one of the reason of the first website of
this kind to bankrupt ( cf History ).


\subparagraph{mediatic issues}

As the video hosting services are delivered by a very few number of
websites ( https://en.wikipedia.org/wiki/List\_of\_video\_hosting\_services
) , even if the internet tends to have a flat topology , the media
seems to not be so far from the TV than its topology suggest. Video
hosting website are not a top- down communication system but it is
not as flat as it is said to be ( source ). Everyone can express itself
but the most viewed videos are made more suggested than the other
ones. it tends to centralize the audience on a few channels and can
lead to a lack diverssity of point of view.


\subparagraph{censorship issues}

With the mirage about diversity of opinions on those platforms, there
is no more freedom of speaking in every country. Being cetralized
make it possbiel to be controlled or pressured by governments. For
Youtube alone, a list of 23 countries for which censorship is applied
is available on wikipedia. It includes Afghanistan, Armenia, Bangladesh,
Brazil, China, Eritrea, Germany, Indonesia, Iran, Libya, Malaysia,
Morocco, North Korea, Pakistan, Russia, Syria, Sudan, South Sudan,
Tajikistan, Thailand, Turkey, Turkmenistan. 

Recently, a case about Youtube appeared on the news pointing out that
the channel ``Philip deFranco'' ( and other creators ) has been
demonetized by a Youtube decision because their video would deem not
to be ``advertiser friendly\textquotedbl{}.

http://betanews.com/2016/09/03/youtube-demonetizes-videos/

https://en.wikipedia.org/wiki/Censorship\_of\_YouTube\#.C2.A0Russia

https://en.wikipedia.org/wiki/Censorship\_by\_Google\#Ungoogleable


\paragraph{copyright issues}

In spite of video hosting services can has control on video availability,
it doesn't prevent the break of copyright by several technics confusing
the robots checking the videos. Even for users it could be confusing
and they may not konw if they are watching a copyrighted video or
not.( sources ) 


\paragraph{Futur usage}

the samrtphones usage growing, the watching of videos is compelled
to increase sensibly and is an important market to reach on the future.
Global mobile phone subscriptions exceeded 2 billion in 2013 and are
expected to reach 8 billion subscriptions by 2019, according to the
Ericsson Mobility Report ( November 2013 ).

http://www.cwcsi.com/smartphones-drives-video-usage-increasing-mobile-data-traffic/


\subsubsection{peer-to-peer}

https://en.wikipedia.org/wiki/Peer-to-peer

The Peer-to-peer (P2P) is a point to point application architecture
where the data ( or the tasks) are distributed among the peers of
the network. Each nodes are equally responsible and of the service
provided and get a fraction of the task or the data to take care of.
They are said to form a peer-to-peer network of nodes.

The Peers are using and serving the resources of the network. They
make a part of their resources network, such as disk storage or network
bandwidth, available for the other peers. In contrast to the traditional
client-server model in which the consumption and supply of resources
is divided. 


\paragraph{History}

While P2P systems had previously been used in many application domains,{[}3{]}
this architecture became famous with the file sharing system : Napster.
The music service, originally released in 1999. The original company
ran into legal difficulties over copyright infringement, ceased operations
and was finnally bought by Roxio then Rhapsody. The original service
is now a music store, but the peer-to-peer concept gave birth to a
lot of following companies and projects like : Gnutella, Freenet,
Kazaa, Bearshare, LimeWire, Scour, Grokster, Madster, and eDonkey2000.
some of them didn't stand until now mainly because of copyright issues. 


\paragraph{Potential}

The concept of Napster has inspired new structures and philosophies
in many areas of human interaction. In such social contexts, peer-to-peer
as a meme refers to the egalitarian social networking that has emerged
throughout society, enabled by Internet technologies in general.

Emerging collaborative P2P systems gives more power and possibilities
to the system a network peers involved in a service than the hability
of a single service to acheive something individually. {[}2{]}


\paragraph{drawbaks}

startup time efficiency / coordination


\subsubsection{streaming websites}

https://en.wikipedia.org/wiki/Streaming\_media

Streaming media is multimedia that is constantly received by and presented
to an end-user while being delivered by a provider. The verb \textquotedbl{}to
stream\textquotedbl{} refers to the process of delivering media in
this manner; the term refers to the delivery method of the medium,
rather than the medium itself, and is an alternative to file downloading.

A client media player can begin to play the data (such as a movie)
before the entire file has been transmitted. Distinguishing delivery
method from the media distributed applies specifically to telecommunications
networks, as most of the delivery systems are either inherently streaming
(e.g. radio, television) or inherently nonstreaming (e.g. books, video
cassettes, audio CDs). For example, in the 1930s, elevator music was
among the earliest popularly available streaming media; nowadays Internet
television is a common form of streamed media. The term \textquotedbl{}streaming
media\textquotedbl{} can apply to media other than video and audio
such as live closed captioning, ticker tape, and real-time text, which
are all considered \textquotedbl{}streaming text\textquotedbl{}. The
term \textquotedbl{}streaming\textquotedbl{} was first used in the
early 1990s as a better description for video on demand on IP networks;
at the time such video was usually referred to as \textquotedbl{}store
and forward video\textquotedbl{},{[}1{]} which was misleading nomenclature.

As of 2016, streaming is generally taken to refer to cases where a
user watches digital video content and/or listens to digital audio
content on a computer screen and speakers (ranging from a desktop
computer to a smartphone) over the Internet. With streaming content,
the user does not have to download the entire digital video or digital
audio file before she starts to watch/listen to it. There are challenges
with streaming content on the Internet. If the user does not have
enough bandwidth in her Internet connection, she may experience stops
in the content and some users may not be able to stream certain content
due to not having compatible computer or software systems. As of 2016,
popular streaming website include YouTube, which contains video and
audio files on a huge range of topics and Netflix, which streams movies
and TV shows.

Live streaming refers to Internet content delivered in real-time,
as events happen, much as live television broadcasts its contents
over the airwaves via a television signal. Live internet streaming
requires a form of source media (e.g. a video camera, an audio interface,
screen capture software), an encoder to digitize the content, a media
publisher, and a content delivery network to distribute and deliver
the content. Live streaming does not need to be recorded at the origination
point, although it frequently is.


\paragraph{History}

In the early 1920s, George O. Squier was granted patents for a system
for the transmission and distribution of signals over electrical lines{[}2{]}
which was the technical basis for what later became Muzak, a technology
streaming continuous music to commercial customers without the use
of radio. Attempts to display media on computers date back to the
earliest days of computing in the mid-20th century. However, little
progress was made for several decades, primarily due to the high cost
and limited capabilities of computer hardware. From the late 1980s
through the 1990s, consumer-grade personal computers became powerful
enough to display various media. The primary technical issues related
to streaming were: having enough CPU power and bus bandwidth to support
the required data rates and creating low-latency interrupt paths in
the operating system to prevent buffer underrun and thus enable skip-free
streaming of the content. However, computer networks were still limited
in the mid-1990s, and audio and video media were usually delivered
over non-streaming channels, such as by downloading a digital file
from a remote server and then saving it to a local drive on the end
user's computer or storing it as a digital file and playing it back
from CD-ROMs.


\paragraph{Bandwidth and storage }

Unicast connections require multiple connections from the same streaming
server even when it streams the same content

A broadband speed of 2 Mbit/s or more is recommended for streaming
standard definition video without experiencing buffering or skips,
especially live video,{[}10{]} for example to a Roku, Apple TV, Google
TV or a Sony TV Blu-ray Disc Player. 5 Mbit/s is recommended for High
Definition content and 9 Mbit/s for Ultra-High Definition content.{[}11{]}
Streaming media storage size is calculated from the streaming bandwidth
and length of the media using the following formula (for a single
user and file) requires a storage size in megabytes which is equal
to length (in seconds) \texttimes{} bit rate (in bit/s) / (8 \texttimes{}
1024 \texttimes{} 1024). For example, one hour of digital video encoded
at 300 kbit/s (this was a typical broadband video in 2005 and it was
usually encoded in a 320 \texttimes{} 240 pixels window size) will
be: (3,600 s \texttimes{} 300,000 bit/s) / (8\texttimes 1024\texttimes 1024)
requires around 128 MB of storage.

If the file is stored on a server for on-demand streaming and this
stream is viewed by 1,000 people at the same time using a Unicast
protocol, the requirement is 300 kbit/s \texttimes{} 1,000 = 300,000
kbit/s = 300 Mbit/s of bandwidth. This is equivalent to around 135
GB per hour. Using a multicast protocol the server sends out only
a single stream that is common to all users. Therefore, such a stream
would only use 300 kbit/s of serving bandwidth. See below for more
information on these protocols. The calculation for live streaming
is similar. Assuming that the seed at the encoder is 500 kbit/s and
if the show lasts for 3 hours with 3,000 viewers, then the calculation
is number of MBs transferred = encoder speed (in bit/s) \texttimes{}
number of seconds \texttimes{} number of viewers / (8{*}1024{*}1024).
The results of this calculation are as follows: number of MBs transferred
= 500 x 1024 (bit/s) \texttimes{} 3 \texttimes{} 3,600 ( = 3 hours)
\texttimes{} 3,000 (number of viewers) / (8{*}1024{*}1024) = 1,977,539
MB


\paragraph{multicast vs unicast}

youtube and netflix are in unicast but they use caches limit the data
repeating.

multicast is very good for for TV like streaming but doesn't fit requirements
of VOD 

https://www.reddit.com/r/networking/comments/2cp356/how\_do\_streaming\_services\_like\_netflix\_and/


\subsubsection{direct download}


\subsubsection{video on demand}


\paragraph{netflix}


\subsubsection{IPTV}


\subsection{Demand}


\subsubsection{Survey}

show about : people need a fast delivery.z


\subsubsection{Internet : A mesh network}

History : 

By nature, the Internet is a mesh network . Each points has an address
to connect, exchange and publish information. This was the case in
the beginning of Internet when the actors were professional ones.
However , today the situation isn't the same anymore. Youtube, a video
hosting website, and Facebook, a social network website hosting videos,
are respectively the 2nd and the 3rd most visited websites in the
world. It means that the mesh network is turning into a convergent
network. ( source : http://www.alexa.com/topsites - Alexa Internet,
Inc. is a California-based company that provides commercial web traffic
data and analytics. It is a wholly owned subsidiary of Amazon.com.
source wikipedia).

It isn't a porblem in itself. The actors of the Internet are acting
according to the economy model we are living in. It could be a problem
in terms of data traffic, data control , privacy control.


\paragraph{A Recentralization of the network}

Unlike the historical Mainframe where all the data where stored and
computed in a central place, the Internet nature is to be a mesh network
: decentralized.

As it is not a problem in itself, concentrate the world wide web in
a few nodes is costly: hundreds of millions of connexions per second
to one node doesn't come without a price. 


\subparagraph{ancient vision of the internet}

https://en.wikipedia.org/wiki/Peer-to-peer

Tim Berners-Lee's vision for the World Wide Web was close to a P2P
network in that it assumed each user of the web would be an active
editor and contributor, creating and linking content to form an interlinked
\textquotedbl{}web\textquotedbl{} of links. The early Internet was
more open than present day, where two machines connected to the Internet
could send packets to each other without firewalls and other security
measures.{[}4{]} This contrasts to the broadcasting-like structure
of the web as it has developed over the years.{[}6{]} As a precursor
to the Internet, ARPANET was a successful client-server network where
\textquotedbl{}every participating node could request and serve content.\textquotedbl{}
However, ARPANET was not self-organized, and it lacked the ability
to \textquotedbl{}provide any means for context or content-based routing
beyond 'simple' address-based routing.\textquotedbl{}{[}7{]}


\subparagraph{bottle-neck network struggling}

http://www.csmonitor.com/Technology/2015/1223/YouTube-says-T-Mobile-is-downgrading-videos.-Does-that-violate-net-neutrality-video 

The amount of data downloaded and uploaded is http://www.slate.fr/story/67161/google-free-video-interconnexion-rame


\subparagraph{Infrastructure: as the big mainframe serving all the videos, YouTube
records five hundred hours of video per minutes, at the time I am
writing the thesis and could reach 700h next year. ( source: http://tubularinsights.com/hours-minute-uploaded-youtube/
). The company can handle the cost of storage and the necessary bandwidth
with advertisement money. }


\subparagraph{Data Control}
\begin{itemize}
\item central point
\item Data owner?
\item youtube algo copyright
\end{itemize}

\subparagraph{market examples}
\begin{itemize}
\item Centralized: 

\begin{itemize}
\item Deezer / Spotify 

\begin{itemize}
\item Youtube / torrents
\end{itemize}
\end{itemize}
\end{itemize}
\begin{figure}[H]
\centering{}\includegraphics[clip,width=0.7\paperwidth,height=0.7\paperwidth]{\string"/home/m/Desktop/Screenshot at 2016-08-23 22:05:17\string".eps}\caption{increasing of Internet users over the year}
\end{figure}
 


\subsubsection{Sum up}

The problem is constituted by the traffic dragged to a few hosting
platforms like YouTube, Facebook and produce a clogging of the network.
Although this one would be more efficient if the hosting of videos
would be spread over the network. 

An other problem risen by the survey is about the speed of watching.
the big advantage of big servers like YouTube lies in the fast delivery
of videos. With a computing power at the level of the companies hosting
these sites , with the means employed there the delivery can be fast.


\subsection{Market}


\section{Technical Context}


\subsection{JPaxos}

JPaxos is a Java library and runtime system for efficient state machine
replication. With JPaxos it is very easy to make a user-provided service
tolerant to machine crashes. Our system supports the crash- recovery
model of failure and tolerates message loss and communication delays.
State machine replication is a general method for implementing a fault-tolerant
service by replicat- ing it on separate machines and coordinating
client interactions with these replicas (or copies). The physical
isolation of machines in a distributed system ensures that failures
of server replicas are independent, as required. As long as there
are enough of non-faulty replicas, the service is guaranteed to be
provided. JPaxos makes the following assumptions about the replicated
service: • deterministic behaviour, i.e. multiple copies of the service
begun in the start state, receiving the same inputs in the same order
will arrive at the same state having generated the same outputs •
non-Byzantine failures, i.e. a service machine can only crash • crash-recovery
supported, i.e. after crash, the service can be restarted with the
same IP address.


\subsection{peer-to-peer}

Peer-to-peer file sharing is the distribution and sharing of digital
media using peer-to-peer (P2P) networking technology. P2P file sharing
allows users to access media files such as books, music, movies, and
games using a P2P software program that searches for other connected
computers on a P2P network to locate the desired content.{[}1{]} The
nodes (peers) of such networks are end-user computers and distribution
servers (not required).

Peer-to-peer file sharing technology has evolved through several design
stages from the early networks like Napster, which popularized the
technology, to the later models like the BitTorrent protocol. Microsoft
uses it for Update distribution (Windows 10) and online playing games
(e.g. the mmorpg Skyforge{[}2{]}) use it as their content distribution
network for downloading large amounts of data without incurring the
dramatic costs for bandwidth inherent when providing just a single
source.

Several factors contributed to the widespread adoption and facilitation
of peer-to-peer file sharing. These included increasing Internet bandwidth,
the widespread digitization of physical media, and the increasing
capabilities of residential personal computers. Users were able to
transfer either one or more files from one computer to another across
the Internet through various file transfer systems and other file-sharing
networks.{[}1{]}


\subsubsection{protocol}


\subsubsection{Projects}


\paragraph{Popcorn Time following projects}


\subparagraph{popcorntimes}

Popcorn Time is a multi-platform, free software BitTorrent client
that includes an integrated media player. The applications provide
a free alternative to subscription-based video streaming services
(such as Netflix). Popcorn Time uses sequential downloading to stream
video listed by several torrent websites (although other trackers
can be added and used manually).

Following its inception, Popcorn Time quickly received positive media
attention, with some comparing the app to Netflix for being easy to
use.{[}5{]} After this increase in popularity, the program was abruptly
taken down by its original developers on March 14, 2014, due to pressure
from the MPAA.{[}6{]} Since then, the program has been forked several
times with several other development teams such as the Butter Project
to maintain the program and produce new features. The original Popcorn
Time team endorsed the popcorntime.io fork, and picked it as the successor
to the official Popcorn Time as of August 2015.{[}7{]} In October
2015, the MPAA obtained a court injunction from Canada to stop the
Canadian programmers of popcorntime.io,{[}8{]} and later obtained
the domain name,{[}9{]} although the project reappeared on a new website
popcorntime.sh.{[}10{]}{[}11{]}{[}12{]}


\subparagraph{Butter project}

Butter Project or simply Butter is a suite of open source desktop
and mobile applications that allow video-streaming over the BitTorrent
protocol.{[}1{]} The project was first made public on the 23rd of
October 2015.{[}2{]} The aim is to create a completely legal base
which other applications can use to provide streaming functionality.{[}2{]}

Butter Project was created as a split from Popcorn Time when the latter
met legal difficulties — with Butter Project aiming to retain development
of only expressly legal and permissible portions of the code-base,
relating to video-streaming. The developers have asserted Butter Project
will not use any of the popcorntime.io infrastructure.{[}3{]} Butter
Project is not aimed at allowing copyright infringement, but aims
to build the groundworks for streaming video over BitTorrent.{[}4{]}
By having a legal portion which remains on GitHub the creators hope
they can get more developers involved.{[}4{]}

By default Butter can play content from VODO which carries free videos,
but will also be configurable to allow for custom sources of video.{[}5{]}


\subsubsection{security}

Security and trust

Peer-to-peer systems pose unique challenges from a computer security
perspective.

Like any other form of software, P2P applications can contain vulnerabilities.
What makes this particularly dangerous for P2P software, however,
is that peer-to-peer applications act as servers as well as clients,
meaning that they can be more vulnerable to remote exploits.{[}31{]}
Routing attacks

Also, since each node plays a role in routing traffic through the
network, malicious users can perform a variety of \textquotedbl{}routing
attacks\textquotedbl{}, or denial of service attacks. Examples of
common routing attacks include \textquotedbl{}incorrect lookup routing\textquotedbl{}
whereby malicious nodes deliberately forward requests incorrectly
or return false results, \textquotedbl{}incorrect routing updates\textquotedbl{}
where malicious nodes corrupt the routing tables of neighboring nodes
by sending them false information, and \textquotedbl{}incorrect routing
network partition\textquotedbl{} where when new nodes are joining
they bootstrap via a malicious node, which places the new node in
a partition of the network that is populated by other malicious nodes.{[}32{]}
Corrupted data and malware See also: Data validation and Malware

The prevalence of malware varies between different peer-to-peer protocols.
Studies analyzing the spread of malware on P2P networks found, for
example, that 63\% of the answered download requests on the Limewire
network contained some form of malware, whereas only 3\% of the content
on OpenFT contained malware. In both cases, the top three most common
types of malware accounted for the large majority of cases (99\% in
Limewire, and 65\% in OpenFT). Another study analyzing traffic on
the Kazaa network found that 15\% of the 500,000 file sample taken
were infected by one or more of the 365 different computer viruses
that were tested for.{[}33{]}

Corrupted data can also be distributed on P2P networks by modifying
files that are already being shared on the network. For example, on
the FastTrack network, the RIAA managed to introduce faked chunks
into downloads and downloaded files (mostly MP3 files). Files infected
with the RIAA virus were unusable afterwards and contained malicious
code. The RIAA is also known to have uploaded fake music and movies
to P2P networks in order to deter illegal file sharing.{[}34{]} Consequently,
the P2P networks of today have seen an enormous increase of their
security and file verification mechanisms. Modern hashing, chunk verification
and different encryption methods have made most networks resistant
to almost any type of attack, even when major parts of the respective
network have been replaced by faked or nonfunctional hosts.{[}35{]}


\subsection{IPFS}

InterPlanetary File System (IPFS) is a content-addressable, peer-to-peer
hypermedia distribution protocol. Nodes in the IPFS network form a
distributed file system. IPFS is an open source project developed
by Protocol Labs with help from the open source community.{[}1{]}
It was initially designed by Juan Benet.{[}2{]} The goal of IPFS is
to facilitate a permanent and decentralized method of storing and
sharing files.{[}3{]}


\subsection{JavaScript technologies}
\begin{itemize}
\item Which tools?

\begin{itemize}
\item Player Torrent: Popcorn Time 
\item Client Torrent web: Deluge 
\item Tracker Torrent: OpenBittorent
\item Distributed Server: JPaxos : concurrent server access manager Data
Replication
\end{itemize}
\item Butter Project: 

\begin{itemize}
\item NodeJS
\item npm 
\item (Angular ) 
\item Grunt 
\item Gulp
\end{itemize}
\end{itemize}

\subsection{Others}


\section{Thesis motivation}


\subsection{Advertisements on the Internet}


\subsubsection{Ads history}


\paragraph{Advertisement definition}


\subparagraph{defeinition}

From Latin, ad vertere : \textquotedbl{}to turn toward\textquotedbl{},
the advertisement is a marketing communication to promote or sell
something, usually a business's product or service. The purpose of
advertisement can also to make sure that employees or shareholders
believe in the company. To be short, the advertisement is a way to
convince a specific public to have a specific behavior ( source wiki
?) : buy a product, elect someone, incitement to reduce speed while
driving, etc. 

the sponsors are usually companies that are looking to improve their
brand image or their sells. It could also be governments in a make
citizen aware of something ( like ecology or safe driving) or political
personalities to raise their popularity. 


\paragraph{from television to the Internet}

If the Internet seems to have conquest most of the planet, the advertisement
remains in a lot of media. Old media such as newspapers, magazines,
Television, Radio, outdoor advertising or direct mail, are still widely
used. But more and more people are using internet and the advertisement
follow.

\begin{table}[] \centering \caption{My caption} \label{my-label} \begin{tabular}{lll} \textbf{media} & \textbf{2015} & \textbf{2016} \\ television & 37,7\% & 34,8\% \\ online & 29,1\% & 36,6\% \\ news & 19,3\% & 15,4\% \\ poster & 6,8\% & 6,6\% \\ Radio & 6,5\% & 5,9\% \\ Cinema & 0,6\% & 0,7\% \end{tabular} \end{table}%%source https://fr.wikipedia.org/wiki/Publicit%C3%A9#cite_note-1

new media such as blogs, websites, or web-videos have recently been
conquered by the sponsors. But bigger implicatinon as to come in the
future as the {[} growing demographic {]}

With the Internet came many new advertising opportunities. Popup,
Flash, banner, Popunder, advergaming, and email advertisements (all
of which are often unwanted or spam in the case of email) are now
commonplace. Particularly since the rise of \textquotedbl{}entertaining\textquotedbl{}
advertising, some people may like an advertisement enough to wish
to watch it later or show a friend.{[}citation needed{]} In general,
the advertising community has not yet made this easy, although some
have used the Internet to widely distribute their ads to anyone willing
to see or hear them. In the last three-quarters of 2009 mobile and
internet advertising grew by 18\% and 9\% respectively. Older media
advertising saw declines: -10.1\% (TV), -11.7\% (radio), -14.8\% (magazines)
and -18.7\% (newspapers). ( source wikipedia ) .

In the past, the most efficient way to deliver a message was to emit
it to the biggest possible public. From now on, advertisers will have
an increasing ability to reach specific audiences. The usage of tracking,
customer profiling and the growing popularity of niche content gives
a a number of more specific and smaller market target. However this
advertisement is more segmented but much better defined ands so it's
much more efficient. With an agreement with a specialist blog writer
in a a specific domain, advertisers are able to comunicate through
famous specialists delivering interresting and and specialised advices
including marketed products to passionates. With this scennario, the
message commuincated by the brand, usually delivered in a top down
topology on tellevisoin, is now certified as a good advice by a specialist
of the area. The impact of the message would much more better on internet
than with television.

Among others, Comcast Spotlight is one such advertiser employing this
method in their video on demand menus. These advertisements are targeted
to a specific group and can be viewed by anyone wishing to find out
more about a particular business or practice, from their home. This
causes the viewer to become proactive and actually choose what advertisements
they want to view.{[}65{]}

Google AdSense is an example of niche marketing. Google calculates
the primary purpose of a website and adjusts ads accordingly; it uses
key words on the page (or even in emails) to find the general ideas
of topics disused and places ads that will most likely be clicked
on by viewers of the email account or website visitors.{[}citation
needed{]}.


\subsubsection{Advertisement Effect on humans}

neuromarketing


\subsubsection{Nowadays, on Internet ( actual state )}


\paragraph{Numbers}


\paragraph{croustillant}


\paragraph{Money spent}

In 2015, the world spent an estimate of US\$592.43 billion on advertising.{[}4{]}
Internationally, the largest (\textquotedbl{}big four\textquotedbl{})
advertising conglomerates are Interpublic, Omnicom, Publicis, and
WPP.{[}5{]} ( https://en.wikipedia.org/wiki/Advertising):

money for 1 ad . for 1 video, etc... 


\subsection{Aletnatives}


\subsubsection{Adblockers}

It became difficult to rbowse the internet without being bombed with
advertising pop ups, frames , etc. In order to prevent ourself from
these enoyements, a set of addons and plugins emerged on each browsers
to stop them : the adblockers ( or content filtering ).

wiki: To users, the benefits of ad blocking include quicker loading
and cleaner looking web pages free from advertisements, lower resource
waste (bandwidth, CPU, memory, etc.), and privacy benefits gained
through the exclusion of the tracking and profiling systems of ad
delivery platforms. Blocking ads can also save substantial amounts
of energy.{[}1{]}{[}2{]}

It is possible for ad-blocking to benefit the natural environment
via an indirect route. This arises because the advertising-and-marketing
industry places a strong emphasis on building emotional connections
with inanimate objects for sale,{[}3{]} {[}4{]} and on creating the
urge to buy immediately.{[}5{]} {[}6{]} With the average person seeing
more than 5000 advertisements every day, and with many of these being
from online sources,{[}7{]} and with each ad promising viewers that
their lives will be improved if they buy what is being promoted, {[}8{]}
{[}9{]} {[}10{]} it is reasonable to expect that some people will
end up buying items which they may not actually need.{[}11{]} If these
items then end up being disposed of without actually being used, then
the environmental impacts of waste disposal will inevitably arise.
Because advertisements are very carefully crafted to target weaknesses
in human psychology,{[}12{]} {[}3{]} it therefore follows that a reduction
in advertisements viewed would result in less waste to dispose of.

Unwanted advertising can also harm the advertisers themselves if the
users become irritated by the ads. Irritated users might make a conscious
effort to avoid the goods and services of firms with annoying ads.{[}13{]}

For users not interested in making purchases, the blocking of ads
can also save time. Any ad that appears on a website exerts a toll
on the user's attention budget, since each ad enters the user's field
of view and must either be consciously ignored or closed, or dealt
with in some other way. A user who is strongly focused on reading
solely the content that he/she is seeking likely has no desire to
be diverted by advertisements that seek to sell unneeded or unwanted
goods and services.{[}14{]} In contrast, users who are actively seeking
items to purchase might appreciate advertising, in particular targeted
ads.{[}15{]}

Ad-blocking can also save money for the user. If a user's personal
time is worth one dollar per minute, and if unsolicited advertising
adds an extra minute to the time that the user requires for reading
the webpage (i.e. the user must manually identify the ads as ads,
and then click to close them, or use other techniques to either deal
with them, all of which tax the user's intellectual focus in some
way),{[}16{]} then the user has effectively lost one dollar of time
in order to deal with ads that might generate a few fractional pennies
of display-ad revenue for the website owner. The problem of lost time
can rapidly spiral out of control if malware accompanies the ads.{[}17{]}{[}18{]}
This is discussed in more detail below.

Another important aspect is improving security; online advertising
subjects users to a higher risk of infecting their devices than surfing
pornography sites.{[}19{]} In a high-profile case, malware was distributed
through advertisements provided to YouTube by a malicious customer
of Google's Doubleclick.{[}20{]}{[}21{]} In August 2015, a 0-day exploit
in the Firefox browser was discovered in an advertisement on a website.{[}22{]}
When Forbes required users to disable ad blocking before viewing their
website, those users were immediately served with pop-under malware.{[}23{]}

Users who pay for total transferred bandwidth (\textquotedbl{}capped\textquotedbl{}
or pay-for-usage connections) including most mobile users worldwide,
have a direct financial benefit from filtering an ad before it is
loaded. Streaming audio and video, even if they are not presented
to the user interface, can rapidly consume gigabytes of transfer especially
on a faster 4G connection. Users without a data plan often pay by
the megabyte, the cost of tolerating ads can be quite high. Even fixed
connections are often subject to usage limits, especially the faster
connections (100Mbit/s and up) which can quickly saturate a network
if filled by streaming media.{[}citation needed{]}

It is a known problem with most web browsers, including Firefox, that
restoring sessions often plays multiple embedded ads at once.{[}24{]}
Using an advertisement blocker stops such behaviour.

--

Popularity 

The number of monthly active adblocking users, as estimated by PageFair.

Use of mobile and desktop ad blocking software designed to remove
traditional advertising grew by 41\% worldwide and by 48\% in the
U.S. between Q2 2014 and Q2 2015.{[}25{]}

As of Q2 2015, 45 million Americans were using ad blockers.{[}25{]}
In a survey research study released Q2 2016, MetaFacts reported 72
million Americans, 12.8 million adults in the UK, and 13.2 million
adults in France were using ad blockers on their PCs, Smartphones,
or Tablets.

In March 2016, the Internet Advertising Bureau reported that UK adblocking
was already at 22\% among over-18s.{[}26{]}{[}27{]}


\subsubsection{Anti-ads movement}

The profiling on internet is became very popular to do adverstisement
. The tracking of activity on internet has began quite profitable
for the announcers. As a result, some public organisation and groups
are criticizing the phenomenon.


\paragraph{Playlist launching}


\paragraph{Non-commercialized video}

Some content producers like Usul2000, even decidedd to 


\paragraph{Anti-Tracking}

duckduckgo


\subsubsection{An other way to finance creators : patronage}

Recentently, an old trend came back on the front of scene :the patronage.
Indeed, it is a way to pay people that produce content on the video
hosting websites. 

In 2013, a few websites emerged with the idea to retribute content
makers like musiciens or comic short movies with tips and live from
that. tipeee .com in France and Patreon in US. 

Their growth is impressive, Tipeee already gathers more than 524 000
€ of tips, while Patreon collected 50 Millions \$ for its creators
members. It is clearly called to continue as People want to support
the free creators and give in order to get nice videos. Everybody
understand that without money, creators cannot continue to provide
freely good content. so there is 2 solutions : accept the ads or tip
the creator.


\chapter{Concept and Design of the System}

In this part, I will try to explain how I will propose an answer to
the problem stated.


\section{Conception Process methodology ( Quality )}


\subsection{scrum}


\subsection{Organisation}


\subsection{trello}


\subsection{pomodoro}


\section{Concept}


\subsection{The market demand}

The expectations are taken from the survey and from the background
study part that explain why these folowing expectations has been choosen.


\subsubsection{No advertisement}

The first of the expectations of the software is to not have advertisement. 


\paragraph{Market growing need}

As it is discussed in the first part, ads have unwanted effects on
watchers and we would like to prevent them from these effects as there
is a a growing demand of not advertised content ( e.g. first part
) . 


\paragraph{a Free service}


\subparagraph{emerge on the market}

We would like to provide a service free for the user. On the market
of video hosting , most of the actors are free of costs. Our solution
would not be attractive if we would ask users to pay. Some actors
like Netflix are not free but they provide copyrighted series inan
easy way that is not our target.

In order to emerge on the market , we would need to be free as Youtube. 


\subparagraph{Not supporting the hosting}

the problem is that providing a free service doesn't get along with
the previous point. Indeed , Youtube finance the hosting cost of its
service with the advertisement. We tried to conciliate both of the
aspect in our project. That's means supporting not supporting the
hosting cost of the video.


\subsubsection{quality service}

As it has been reported on the survey, one important aspect of a video
hosting website is the speed delivery and quality of the video. In
that sens we will try to optimze as its best the speed of delivering
and the quality of the supplyed videos.


\subsubsection{Simple}

The product should be easy to use. The user interface is most important
thing in a such application ...


\subsubsection{Safe / a respectful service}

Obviously the application should be secure. It implies not possiblity
to track people or to be able to get personnal data throught it.
\begin{itemize}
\item Privacy
\item no ads needed
\item Distributed video platform: 

\begin{itemize}
\item Computer network ( Skype , torrents, popcorntime ) 
\item Supply Videos : access speed 
\item Distribution over the network Manage video samples Central Website
\end{itemize}
\end{itemize}

\subsubsection{an Internet compatible service}

One of the objectives of this project is to respect the mesh network
philosophy. We would like to make prevailing a flat consumption of
the internet. It means that it should propose a decentralization of
the Internet as it has been made for at the origin.


\subsubsection{a reliable service}


\paragraph{target fragile networks}

The aim of this project is also to give access to video hosting enjoyment
to the poorly connected areas. In a lot of places in the world like
India or Africa, the internet connection is not a certitude. 


\paragraph{failure resistant}

We would like our software to be as much failure resistant as possible. 


\paragraph{scalability}

The software should also be prepared to absorb a big growth of users
without crashing of slow the serving. It would a big advantage compared
to a lot of online streaming website that deliver content very slowly
on the critical hours.
\begin{itemize}
\item Invert growth paradigm: 

\begin{itemize}
\item More People → More Fun 
\item No Internet? No problem.. ( Africa , parts of the web where efficiency
matters)
\end{itemize}
\end{itemize}

\subsection{Marketing answer}


\subsubsection{An emerging market}


\subsubsection{An other way to see views}


\subsubsection{Economic plan}


\subsection{The technical answer}


\subsubsection{User supported hosting}

In order to be able to emerge on the market , it is necessary to opt
for a free service for the users. A paid service would have too much
difficulty to get known. However we should take into consideration
the cost of hosting files. As we want a free service, we cannot invest
in servers. We can't rely on money entries generated by advertisement
either. The hosting of the service has to be powered mostly by the
users. 


\subsubsection{Network management }

a way to ensure the speed of high definition videos and the best repartition
of the videos among the network. We may use optimization algorithms.


\subsubsection{Video player}

To display the content of the video, we need a video downloaded and
a video player that would be able to play the video as it is not finished
to load yet. 


\subsubsection{Quality content }

We will try to get the best quality content.


\section{The Design}

In this part, I will expose the design architecture


\subsection{tools review}

To build the application , I will connect several tools.


\subsubsection{Decentralized hosting, ad-free, high quality videos, reliable}

The main tool is used ot build the application is the peer-to-peer
protocol. It answer a lot problems I tried to solve. 


\paragraph{Advantages}

It allow users to host the files they are viewing. 

There is a variety of softwares implementing it in any languages,
usually free and open-source. most of the bittorrent client doesn't
has any advertisement.

It is reliable : it is used by more than 10 years. We already know
the problems it has, the drawbacks and the advantages.

The way to provide torrent files has been discribed by its createor
( bittorrent.com) and it consits in serving a website to host a the
torrent files with a page of description for each file. Then you need
to open the file with a client supporting the torrent protocol, wait
for the download and finnally be able to play it.


\paragraph{Drawbacks}

This way was convenient in the past but nowdays , with the actual
bandwidth, we are used to get the content faster. A lot of serving
file website arrived. Even if it is senseless due to the traffic it
generate to a few nodes of the network.

Today, according to the survey and wathcing the market, people want
to check videos and directly watch it. So, it needs an overlayer to
improve the torrent protocol to get faster.


\paragraph{tool used}

For the needs of this project, I decided to use Tturns Bittorrent
library ( source ) . It is a minimalist bittorrent client and software
library that just do its job and not more.


\subsubsection{filling the gap of the torrent }

As only a torrent client would not be sufficient, it is neceessary
to improve the torrent by adding other tools to 


\paragraph{Network management}

In order to improve the download time due to the torrent procol. My
idea was to increase the number of copies of a file if it doesn't
have enough and ( maybe to decrease the number of copies. I chose
to manage the distribution of the torrent files over the network.
I would like to make more available the begining of each video file
in order to improve availlability of this and by the same manneer
, the speed of download.

JPAXOS def?


\paragraph{Instant Video player}

Also , I needed a video player that would be able to play videos without
the need of the video to be full. I found butter project that is a
similar project but not focused on the network management. I based
my development on this tool as a module in oreder to make a modular
tool available to everything Torrent client.

Butter Project

Butter Project or simply Butter is a suite of open source desktop
and mobile applications that allow video-streaming over the BitTorrent
protocol.{[}1{]} The project was first made public on the 23rd of
October 2015.{[}2{]} The aim is to create a completely legal base
which other applications can use to provide streaming functionality.{[}2{]}

Butter Project was created as a split from Popcorn Time when the latter
met legal difficulties — with Butter Project aiming to retain development
of only expressly legal and permissible portions of the code-base,
relating to video-streaming. The developers have asserted Butter Project
will not use any of the popcorntime.io infrastructure.{[}3{]} Butter
Project is not aimed at allowing copyright infringement, but aims
to build the groundworks for streaming video over BitTorrent.{[}4{]}
By having a legal portion which remains on GitHub the creators hope
they can get more developers involved.{[}4{]}

By default Butter can play content from VODO which carries free videos,
but will also be configurable to allow for custom sources of video.{[}5{]}


\paragraph{video sources}
\begin{itemize}
\item OpenBittorent OpenTrackers.org Vodo.net Archive.org CCC Media
\item my personnal network
\end{itemize}

\subsection{Architecture}

In this part I expose how the system has been imagined and how it
works.

\begin{wrapfigure}{o}{0.5\columnwidth}%
\includegraphics[width=0.8\paperwidth]{images/DesignSchemaLatex1}

\caption{application schema}
\end{wrapfigure}%



\subsubsection{Normal usage}

Usually, to get a file via torrents, the user should open a torrent
file with his client application. Then the torrent get downloaded
in a download folder.


\subsubsection{Application principle }


\paragraph{The client}

The application need to be configured with the folder containing the
torrents downloaded and the folder containing the downloaded files.
The Client starts listing all the torrent files and compare them with
the related downloaded files it finds. So that , it can figure out
what are the hosted files and send the related torrents files to one
of the JPaxos servers instances. 

It will 


\paragraph{JPaxos server}


\chapter{Implementation}

During the implementation part , I tried to maximize the user experience
and the usability of the application. I will first speak about the
architecture of the application : I tried to keep it as simple as
possible. Then I will detail the code implementation to show how the
application works.




\section{The Client}


\subsection{KaClient}


\subsubsection{Purpose}

The client class is the main of the application. Its function is to
give periodical reports of the hosting content of the Client to the
JPaxos server. It also has a composition with the bittorrentClient
class that is in charge of downloading and sharing torrents. 


\subsubsection{Process}

Here is a presentation of the actual Client class. 
\begin{enumerate}
\item The client start by the traditional main static method
\item then it launches a thread method ``run'' where the main client code
is executed.
\item an infinite loop doing a procedural process and finishing by a sleep()
timer in order to re-execute the loop periodically ( every 10 minutes
);
\end{enumerate}

\subsubsection{UML Diagram}

\begin{figure}[H]
\begin{centering}
\includegraphics{\string"images/UML Diagrams/KaClient\string".eps}
\par\end{centering}

\caption{KaClient UML Diagram}
\end{figure}



\subsubsection{Code details}


\paragraph{The main static method}

\inputencoding{latin9}\begin{lstlisting}
public static void main(String[] args) throws IOException, ReplicationException, InterruptedException, ClassNotFoundException, NoSuchAlgorithmException {
        instructions(); 
        KaClient client = new KaClient(); //class instantiation 
        client.run();    // launch the thread
}
\end{lstlisting}
\inputencoding{utf8}


\paragraph{The thread initialization}

\inputencoding{latin9}\begin{lstlisting}
//initialize the torrent manager
KaTorrentManager tm = new KaTorrentManager(new File("torrents"), 
		new File("downloads"));

//connect to the client         
Client paxosClient = new Client();
paxosClient.connect();
\end{lstlisting}
\inputencoding{utf8}


\paragraph{The infinite loop}

\inputencoding{latin9}\begin{lstlisting}[language=Java,float,numbers=left,breaklines=true,tabsize=4]
while (true) {
            //send data downloads             
			ArrayList<?> list = tm.downloadsStateToArray();

			System.out.println(String.format("DATA SENT \n"+list.toString()));
			KaCommand command = new KaCommand("String", list);
            
			// send     : added /deleted             
			byte[] response = paxosClient.execute(command.toByteArray());
 
            // receive  : missing torrents
		    ByteArrayInputStream bais = new ByteArrayInputStream(response);
            ObjectInputStream ois = new ObjectInputStream(bais);
	        list = (ArrayList<?>) ois.readObject();
            System.out.println(String.format("Previous value :\n %s", list.toString()));

            // send     : the missing torrents files
            list = tm.getMissingTorrentsFiles((ArrayList<String>) list);
            System.out.println(String.format("DATA SENT \n"+list.toString()));
            command = new KaCommand("File", list);
            response = paxosClient.execute(command.toByteArray());

            // receive  : new torrents + pieces to dl
            ois = new ObjectInputStream(new ByteArrayInputStream(response));
            list = (ArrayList<?>) ois.readObject();
            System.out.println(String.format("Previous value :\n %s", list.toString()));
     
            //download them
            BittorrentClient bc = new BittorrentClient((ArrayList<File>) list, tm.getDownloadsDir(), 604800);

            //sleep             
			Thread.sleep(5000);//would be set to 10 minutes after tests 
       }
\end{lstlisting}
\inputencoding{utf8}


\subsubsection{Code Details}


\paragraph{gather added and deleted downloaded files }

\inputencoding{latin9}\begin{lstlisting}
			//send data downloads             
			ArrayList<?> list = tm.downloadsStateToArray();

			System.out.println(String.format("DATA SENT \n"+list.toString()));
			KaCommand command = new KaCommand("String", list);
\end{lstlisting}
\inputencoding{utf8}


\paragraph{send them to the server}

\inputencoding{latin9}\begin{lstlisting}
			// send     : added /deleted             
			byte[] response = paxosClient.execute(command.toByteArray());
\end{lstlisting}
\inputencoding{utf8}


\paragraph{get the list of torrent missing torrent files }

\inputencoding{latin9}\begin{lstlisting}
			// send     : added /deleted             
			byte[] response = paxosClient.execute(command.toByteArray());
 
            // receive  : missing torrents
		    ByteArrayInputStream bais = new ByteArrayInputStream(response);
            ObjectInputStream ois = new ObjectInputStream(bais);
	        list = (ArrayList<?>) ois.readObject();
            System.out.println(String.format("Previous value :\n %s", list.toString()));
\end{lstlisting}
\inputencoding{utf8}


\paragraph{send the torrent files missing on the server}

\inputencoding{latin9}\begin{lstlisting}
			// send     : the missing torrents files
            list = tm.getMissingTorrentsFiles((ArrayList<String>) list);
            System.out.println(String.format("DATA SENT \n"+list.toString()));
            command = new KaCommand("File", list);
            response = paxosClient.execute(command.toByteArray());
\end{lstlisting}
\inputencoding{utf8}


\paragraph{get the torrents to download}

\inputencoding{latin9}\begin{lstlisting}
			// receive  : new torrents + pieces to dl
            ois = new ObjectInputStream(new ByteArrayInputStream(response));
            list = (ArrayList<?>) ois.readObject();
            System.out.println(String.format("Previous value :\n %s", list.toString()));
\end{lstlisting}
\inputencoding{utf8}


\paragraph{launch the download}

\inputencoding{latin9}\begin{lstlisting}
			//download them
            BittorrentClient bc = new BittorrentClient((ArrayList<File>) list, tm.getDownloadsDir(), 604800);
\end{lstlisting}
\inputencoding{utf8}


\paragraph{sleep}

\inputencoding{latin9}\begin{lstlisting}
 //sleep             
			Thread.sleep(5000);//would be set to 10 minutes after tests 
\end{lstlisting}
\inputencoding{utf8}


\subsection{KaCommand}


\subsubsection{Purpose}

This class execute the command to be serialized and sent to the server.
This class is used on the server and the client side. On the client
side, the main methods used are the constructor with plain parameters
and the method toByteArray that serialize the parameters


\subsubsection{Process}

The KaCommand class is initialized by the KaClient thread with two
parameters. The first is a String ``type'' giving the type of data
the command to process. the second is an ArrayList of the type ``type''
it is used to send first, a list of strings containing the list of
the torrent titles, then the list of Torrent files to send to server
or to download.

Then, the method ``toByteArray'' serialize the two parameters.


\subsubsection{UML Diagram}

\begin{figure}[H]
\begin{centering}
\includegraphics{\string"images/UML Diagrams/KaCommand\string".eps}
\par\end{centering}

\caption{KaCommand UML Diagram}
\end{figure}



\subsubsection{Code Details}


\paragraph{constructor}

\inputencoding{latin9}\begin{lstlisting}
public KaCommand(String type, ArrayList<?> list) {
        this.type = type;
        this.list = list;
    }
\end{lstlisting}
\inputencoding{utf8}


\paragraph{serialization}

\inputencoding{latin9}\begin{lstlisting}
public byte[] toByteArray() throws IOException {
        System.out.println("serialization");      
        ByteArrayOutputStream baos = new ByteArrayOutputStream();
        ObjectOutputStream oos = new ObjectOutputStream(baos);
        oos.writeUTF(getType());
        oos.writeObject(getList());
        return baos.toByteArray();
    }
\end{lstlisting}
\inputencoding{utf8}


\subsection{KaTorrentManager}


\subsubsection{Purpose}

This class Manage and gather information about on hosted,recently
added or deleted torrents.


\subsubsection{Process}

This torrent manager is instantiate by KaClient thread with two parameters
: the folder containing the torrent files used by the original BitTorrentClient
and the folder containing the downloaded files.

This class is used first to get the hosted , recently added or deleted
files related with the the torrent files in the the torrent file foder.

Then, it is used to list the torrent files miising on the server. 

In the end, the torrent manager just gives the Download for BitTorrentClient.


\subsubsection{UML}

\begin{figure}[H]
\begin{centering}
\includegraphics{\string"images/UML Diagrams/KaTorrentManager\string".eps}
\par\end{centering}

\caption{KaTorrentManager UML Diagram}


\end{figure}



\subsubsection{get the downloads State}

\inputencoding{latin9}\begin{lstlisting}
	/**
     * @return added / deleted / hosted list in a single Array
     */
    public ArrayList<String> downloadsStateToArray() {
        updateChanges();

        ArrayList<String> returnedList = new ArrayList<>();

        returnedList.add("added");
        returnedList.addAll(this.addedDownloads);

        returnedList.add("deletedList");
        returnedList.addAll(deletedDownloads);

        return returnedList;
    }
\end{lstlisting}
\inputencoding{utf8}


\paragraph{updates the changes}

\inputencoding{latin9}\begin{lstlisting}
    /**
     * store  added / deleted Downloads
     * then update the current list of HostedDownloads
     */
    private void updateChanges() {
        this.addedDownloads = addedDownloads();
        this.deletedDownloads = deletedDownloads();
        updateLastHostedDownloads();
    }
\end{lstlisting}
\inputencoding{utf8}


\paragraph{hosted Downloads}

\inputencoding{latin9}\begin{lstlisting}
    /**
     * list Configured Download and .torrent Directories to identify matches
     *
     * @return list of hosted Downloads compared with what contains the .torrent file folder configured
     */
    public ArrayList<String> hostedDownloads() {
        // list torrentDir
        torrentsList = filePathsToFileNamesArray(listDir(this.torrentsDir, ".torrent"),".torrent");

        //  list recursively downloadDir depth :2
        downloadsList = filePathsToFileNamesArray(listDir(this.downloadsDir, ""),"");

        //  match hosted torrents
        ArrayList<String> hostedTorrents = (ArrayList<String>) torrentsList.clone();
        hostedTorrents.retainAll(downloadsList);

        return hostedTorrents;
    }
\end{lstlisting}
\inputencoding{utf8}


\paragraph{added Downloads}

\inputencoding{latin9}\begin{lstlisting}
	/**
     *
     * @return added downloads filenames from the last check
     */
    public ArrayList<String> addedDownloads() {
        // for 1 st launch
        if (this.lastHostedDownloads.isEmpty()) {
            return hostedDownloads();
        }

        //we get the lastHostedDownloads - deletedDownloads
        ArrayList<String> keptDownloads = (ArrayList<String>) this.lastHostedDownloads.clone();
        keptDownloads.removeAll(deletedDownloads());

        //we do downloadsList - keptDownloads
        ArrayList<String> addedDownloadsList = hostedDownloads();
        addedDownloadsList.removeAll(keptDownloads);

       return addedDownloadsList;
    }
\end{lstlisting}
\inputencoding{utf8}


\paragraph{Deleted downloads}

\inputencoding{latin9}\begin{lstlisting}
/**
     *
     * @return deleted downloads filenames from the last check
     */
    public ArrayList<String> deletedDownloads() {
        ArrayList<String> deletedDownloads = new ArrayList<>();

        // for 1 st launch
        if (this.lastHostedDownloads.isEmpty()) {
            return deletedDownloads;
        }

        deletedDownloads = getLastHostedDownloads();
        deletedDownloads.removeAll(hostedDownloads());
        return deletedDownloads;
    }
\end{lstlisting}
\inputencoding{utf8}


\subsubsection{get Missing Torrents Files}

\inputencoding{latin9}\begin{lstlisting}
    /**
     * Simply returns a list of .torrents files matching matching a list of filename containing the missing torrents on the server
     * @param list of files names sent by the server containing the filenames it does't have the corresponding .torrent file
     * @return a file list containing the .torrent files not present on the server
     */
    public ArrayList<File> getMissingTorrentsFiles(ArrayList<String> list){
        File[] paths = listDir(torrentsDir,".torrent");
       ArrayList<File> missingTorrents = new ArrayList<>();
        for(File f : paths){
           if(list.contains(f.getName()))
               missingTorrents.add(f);
        }
        return missingTorrents;
    }
\end{lstlisting}
\inputencoding{utf8}


\paragraph{list directory}

\inputencoding{latin9}\begin{lstlisting}
    /**
     * List files in a directory according to an extenion file ( ".torrent"  or nothing for downloads)
     *
     * @param dir
     * @param extension
     * @return liste of filenames ( without .torrent if necessary)
     */
    public File[] listDir(File dir, final String extension) {
        File[] paths;

        // create new filename filter
        FilenameFilter fileNameFilter = (dir1, name) -> {

            // for downloadsDir
            if (extension.equals("")) {
                return true;
            }

            //for torrentsDir
            if (name.lastIndexOf('.') > 0) {
                // get last index for '.' char
                int lastIndex = name.lastIndexOf('.');

                // get extension
                String str = name.substring(lastIndex);

                // match path name extension
                if (str.equals(extension)) {
                    return true;
                }
            }
            return false;
        };

        // returns pathnames for files and directory
        paths = dir.listFiles(fileNameFilter);

        // Paths arrays to SimpleName String List
        return paths;
    }
\end{lstlisting}
\inputencoding{utf8}


\paragraph{Changes full paths array into torrent titles ArrayList}

\inputencoding{latin9}\begin{lstlisting}
    /**
     * this method just turn an a String[] Array containing the full path of each torrents into am ArrayList<String>
     * containing only the titles of the torrent
     * @param paths
     * @param extension
     * @return
     */
    private ArrayList<String> filePathsToFileNamesArray(File[] paths, String extension) {
        ArrayList<String> fileNameList = new ArrayList<String>();

        if (extension.equals("")) {
            for (File path : paths) {
                fileNameList.add((path.getName()));
                if (path.isDirectory())
                    fileNameList.addAll(filePathsToFileNamesArray(listDir(path, ""),""));
            }
        } else {
            for (File path : paths)
                fileNameList.add((path.getName().split(extension))[0]);
        }
        return fileNameList;
    }
\end{lstlisting}
\inputencoding{utf8}


\subsection{BitTorrentClient}


\subsubsection{Purpose}

This class aim to download a list of torrents , and share them.


\subsubsection{Process}


\subsubsection{Code parts}

\begin{figure}[H]
\begin{centering}
\includegraphics{\string"images/UML Diagrams/BittorrentClient\string".eps}
\par\end{centering}

\caption{BitTorrentClient UML Diagram}
\end{figure}



\subsubsection{Code Details}


\paragraph{Constructor}

\inputencoding{latin9}\begin{lstlisting}
    public BittorrentClient(ArrayList<File> list, File downloadDir, int sharingTime) throws IOException, NoSuchAlgorithmException {

        for( File f : list){
            initiateClient(f,downloadDir, sharingTime);
        }

    }
\end{lstlisting}
\inputencoding{utf8}


\paragraph{Initiate the download and share for a torrent File}

\inputencoding{latin9}\begin{lstlisting}
 public void initiateClient(File torrentFile, File downloadDir,int sharingTime) throws IOException, NoSuchAlgorithmException {
        // First, instantiate the Client object.
        Client client = new Client(
                // This is the interface the client will listen on (you might need something
                InetAddress.getLocalHost(),

                // Load the torrent from the torrent file and use the given
                // output directory. Partials downloads are automatically recovered.
                SharedTorrent.fromFile(torrentFile,downloadDir));

        // You can optionally set download/upload rate limits
        // in kB/second. Setting a limit to 0.0 disables rate
        // limits.
        client.setMaxDownloadRate(50.0);
        client.setMaxUploadRate(50.0);

        // At this point, can you either call download() to download the torrent and
        // stop immediately after...
        client.download();

        // Or call client.share(...) with a seed time in seconds:
         client.share(sharingTime);
        // Which would seed the torrent for an hour after the download is complete.
        // Downloading and seeding is done in background threads.
        // To wait for this process to finish, call:
//        client.waitForCompletion();

        // At any time you can call client.stop() to interrupt the download.
    }
\end{lstlisting}
\inputencoding{utf8}


\section{Server Side}


\subsection{KaServer}


\subsubsection{Purpose}

The purpose of the server is to store the torrent file of each torrent
on the network having the application and make statistics about the
torrent popularity.

It is based on Paxos algorithm ( \dcsemph{cf annexes} ) . 


\subsubsection{Process}

In order to work with the Paxos algorithm, at least three intances
of this classes should be launched with the number of the instance
in parameter. 

The server starts recovery from the other intances of the class, the
begin to listen to the server connections.


\subsubsection{UML Diagram}

\begin{figure}[H]
\begin{centering}
\includegraphics{\string"images/UML Diagrams/KaServer\string".eps}
\par\end{centering}

\caption{KaServer UML Diagram}
\end{figure}



\subsubsection{Code Details}

\inputencoding{latin9}\begin{lstlisting}
public static void main(String[] args) throws IOException, InterruptedException,
            ExecutionException {
        if (args.length != 1) {
            usage();
            System.exit(1);
        }
        int localId = Integer.parseInt(args[0]);
        Configuration process = new Configuration();

        Replica replica = new Replica(process, localId, new KaService());

        replica.start();
        System.in.read();
        System.exit(-1);
    }
\end{lstlisting}
\inputencoding{utf8}


\subsection{KaService}


\subsubsection{Purpose}

The Service process the answer receive by the server from a client.


\subsubsection{Process}

The Service deserialize the Client's command containing data. It process
the data received then reply the process result. 


\subsubsection{UML Diagram}

\begin{figure}[H]
\begin{centering}
\includegraphics{\string"images/UML Diagrams/KaService\string".eps}
\par\end{centering}

\caption{KaService UML Diagram}
\end{figure}



\subsubsection{Code Details}


\paragraph{Deserialise the client command}

In the execute method, called by the Server ( KaServer class ) :

\inputencoding{latin9}\begin{lstlisting}
// Deserialise the client command
        KaCommand command = null;
        try {
            command = new KaCommand(value);
        } catch (IOException e) {
            logger.log(Level.WARNING, "Incorrect request", e);
            return null;
        } catch (ClassNotFoundException e) {
            e.printStackTrace();
            logger.log(Level.WARNING, "ClassNotFoundException in request Object ", e);
        }
\end{lstlisting}
\inputencoding{utf8}


\paragraph{title list processing}

\inputencoding{latin9}\begin{lstlisting}
		ByteArrayOutputStream byteArrayOutput = new ByteArrayOutputStream();
        ArrayList<?> objectSent;

		// Processing data
        if (command.getType().equals("String")) {
            getLists((ArrayList<String>) command.getList());
            objectSent = this.missingTorrents;

            //we serialise back the object
            try {
                ObjectOutputStream oos = new ObjectOutputStream(byteArrayOutput);
                oos.writeObject(objectSent);
            } catch (IOException e) {
                e.printStackTrace();
                return null;
            }
        } 
\end{lstlisting}
\inputencoding{utf8}


\paragraph{file list processing}

\inputencoding{latin9}\begin{lstlisting}
// [...] 
else   // command type is File
        {
            //register all missing torrents into the torrent file list
            this.allTorrentsFiles.addAll((ArrayList<File>) command.getList());

            //make average rank
            double av = averageRank();

            //for each torrent that is not good rank
            ArrayList<String> underRankTorrents = getUnderRankTorrents(av);

            ArrayList<File> fileSent = new ArrayList<>();

            for (File f : this.allTorrentsFiles) {
                if (underRankTorrents.contains(f.getName()))
                    fileSent.add(f);
            }
\end{lstlisting}
\inputencoding{utf8}


\paragraph{reserialize and the reply}

\inputencoding{latin9}\begin{lstlisting}
 		   //reserialization of the rely
            try {
                ObjectOutputStream oos = new ObjectOutputStream(byteArrayOutput);
                oos.writeObject(fileSent);
            } catch (IOException e) {
                e.printStackTrace();
                return null;
            }//send the torrents files relates
\end{lstlisting}
\inputencoding{utf8}


\subsection{KaCommand}


\subsubsection{Purpose}

On the Server side, the command sent by the client is deserialized
in order to process the data received. This class is used on the server
and the client side.


\subsubsection{Process}

The KaCommand class is initialized by the KaService. On the server
side, the main method used is the constructor. It takes one byte array
parameters in order to deserialize it and retreive the list and the
type of list it is. 


\subsubsection{UML Diagram}

\begin{figure}[H]
\begin{centering}
\includegraphics{\string"images/UML Diagrams/KaCommand\string".eps}
\par\end{centering}

\caption{KaCommand UML Diagram}
\end{figure}



\subsubsection{Code Details}


\paragraph{constructor and deserialization}

\inputencoding{latin9}\begin{lstlisting}
public KaCommand(byte[] bytes) throws IOException, ClassNotFoundException {
        // read from byte array
        ByteArrayInputStream bais = new ByteArrayInputStream(bytes);
        ObjectInputStream ios = new ObjectInputStream(bais);
        type = ios.readUTF();
        list = (ArrayList<?>) ios.readObject();
    }
\end{lstlisting}
\inputencoding{utf8}


\section{Tests}


\subsection{KaTorrentManagerTest}


\subsubsection{Purpose}

This test class has been done to make sure there is no regression
in the code and to facilitaite the development. The conception of
the KaTorrentManager has been hazardous on its beginings as the concept
of torrent file and downloaded ( torrent ) file are near and confusingly
similar in the common language in spite of their purpose differ greatly.
Moreover the concept of this class changed few time in order to make
efficient statistics.

This test has been created to unsure that the methods are fulfilling
their duty the code.


\subsubsection{Process}


\subsubsection{UML Diagram}

\begin{figure}[H]
\begin{centering}
\includegraphics{\string"images/UML Diagrams/KaTorrentManagerTest\string".eps}
\par\end{centering}

\caption{KaTorrentManagerTest UML Diagram}
\end{figure}



\subsubsection{Code Details}


\paragraph{Setup}

\inputencoding{latin9}\begin{lstlisting}
	@Before
    public void setUp() throws Exception {
        tm = new KaTorrentManager(new File("/home/m/documents/put/s3/MTh/Paxos/JPaxos/src/lsr/testResources/torrents"),
                new File("/home/m/documents/put/s3/MTh/Paxos/JPaxos/src/lsr/testResources/downloads"));

        tm.setLastHostedDownloads(tm.hostedDownloads());

        verifyedList = new ArrayList<String>(Arrays.asList("Pirate Informatique N30 - Aout-Octobre 2016.pdf",
                "Votre sant� par les jus frais de l�gumes et de fruits - Norman Walker.pdf",
                "(pdf+epub) Comment d�coder les gestes de vos interlocuteurs  David Cohen",
                "Le Temps est assassin - Michel Bussi (2016) [ePub].epub"));
    }
\end{lstlisting}
\inputencoding{utf8}


\paragraph{setHostedDownloads Test}

\inputencoding{latin9}\begin{lstlisting}
    @Test
    public void setHostedDownloadsTest() throws Exception {
        assertEquals(verifyedList, this.tm.getLastHostedDownloads());

        }
\end{lstlisting}
\inputencoding{utf8}


\paragraph{getDeletedDownloads Test}

\inputencoding{latin9}\begin{lstlisting}
    @Test
    public void getDeletedDownloadsTest() throws Exception {
        //emptytest
        tm.setLastHostedDownloads(new ArrayList<String>());
        assertEquals(new ArrayList<>(),tm.deletedDownloads());

        //normal test
        tm.setLastHostedDownloads(tm.hostedDownloads());
        ArrayList<String> tmpList = tm.hostedDownloads();
        tmpList.add(0,"new DL");
        tm.setLastHostedDownloads(tmpList);

        assertEquals(tmpList.get(0), tm.deletedDownloads().get(0));

    }
\end{lstlisting}
\inputencoding{utf8}


\paragraph{getAddedDownloads Test}

\inputencoding{latin9}\begin{lstlisting}
    @Test
    public void getAddedDownloadsTest() throws Exception {
         //emptytest
        tm.setLastHostedDownloads(tm.hostedDownloads());
        ArrayList<String> tmpList = tm.hostedDownloads();
        tm.setLastHostedDownloads(new ArrayList<String>());

        assertEquals(tmpList,tm.addedDownloads());

        //normal test
        tm.updateLastHostedDownloads();
        String verify = tmpList.remove(0);
        tm.setLastHostedDownloads(tmpList);

        assertEquals(verify, tm.addedDownloads().get(0));
    }
\end{lstlisting}
\inputencoding{utf8}


\chapter{Performance Evaluation}

tests and optimization .

example

expectations

algo simulations 


\chapter{Conclusions}

General sum up : Context , problem , etc.

The move of audience from TV to the Internet result in a concentration
of the most of the video content in a few nodes of the network. As
Internet is a mesh network, this centralization is detrimental and
against nature.

My Thesis answer the PB:

Thesis specific sum-up 

thesis conclusion sentence

Opening: future of the project , improvements..

\appendix

\chapter{Users Guide}

\backmatter
\begin{thebibliography}{Odnośniki}
\bibitem{sop}A.~Tanenbaum. \emph{Operating Systems Design and Implementation}.
Prentice Hall, 2006.

\bibitem{shareyourworld.com} http://www.beet.tv/2007/07/first-video-sha.html

\bibitem{key-9} Aichner, T. and Jacob, F. (March 2015). \textquotedbl{}Measuring
the Degree of Corporate Social Media Use\textquotedbl{}. International
Journal of Market Research. 57 (2): 257–275.\end{thebibliography}

\end{document}
